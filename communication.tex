\section{Outils de Communication}

\subsection{Notions générales}

\begin{frame}[fragile]{Netiquette}
\begin{itemize}[<+->]
 \item Règles de bonne conduite sur les Internets
 \item Network Etiquette
 \item RFC1855~: \url{http://fgouget.free.fr/netiquette/rfc1855-fr.html}
 \item Communications un à un (mail, messagerie instantanée)
 \item Communications un à plusieurs (listes de diffusion, forums)
 \item Forme
 \begin{itemize}[<+->]
  \item Écrire en entier (pas de langage SMS)~;
  \item Écrire en minuscules, ÉCRIRE EN MAJUSCULES REVIENT À CRIER~;
  \item Réponses sous questions~;
  \item Ne pas dépasser 65 charactères dans un mail~;
  \item Etc.
 \end{itemize}
 \item Fond
 \begin{itemize}[<+->]
  \item Politesse
  \item Respect
 \end{itemize}
\end{itemize}
\end{frame}

\begin{frame}{Types de communication}
 \begin{itemize}[<+->]
  \item Interne~:
  \begin{itemize}[<+->]
   \item Entre les membres du projets, et les contributeurs~;
   \item Liés à l'évolution du projet.
  \end{itemize}
  \item Externe/publique~:
  \begin{itemize}
   \item Nouvelles du projet (versions, évolutions importantes).
  \end{itemize}
 \end{itemize}
\end{frame}

\begin{frame}{Modes de communication}
 \begin{itemize}[<+->]
  \item Synchrones (messagerie instantanée)
  \begin{itemize}[<+->]
   \item IRC
   \item ICQ
   \item Aol Instant Messaging
   \item MSN Messenger
   \item XMPP
   \item VoIP
  \end{itemize}
  \item Asynchrones
  \begin{itemize}[<+->]
   \item Usenet/Newsgroups
   \item Mail
   \item Forums
  \end{itemize}
 \end{itemize}
\end{frame}

\subsection{Communication synchrone}

\begin{frame}[fragile]{Internet Relay Chat}
\framesubtitle{Informations générales}
\begin{itemize}[<+->]
 \item RFC 1459 (août 1988), puis RFC 2810 à 2813 (2000)~;
 \item Client-Serveur~;
 \item Réseaux (plusieurs serveurs connectés entre eux)~;
 \item Discussion orientée salon de discussion (canal, channel, chan)~;
 \item \url{\#canal@reseau}~;
 \item \url{utilisateur@reseau}~;
 \item Modes utilisateur : Opérateur du réseau (o), invisible (i), \dots
 \item Modes~: Opérateur (o), voice (v), Secret (s), \dots
 \item Canal et pseudos volatiles.
\end{itemize}
\end{frame}

\begin{frame}[fragile]{Internet Relay Chat}
\framesubtitle{Réseaux connus}
\begin{itemize}[<+->]
 \item IRCnet~: \url{http://www.ircnet.org}
 \item DALnet~: \url{http://www.dal.net}
 \item EFnet~: \url{http://www.efnet.org}
 \item Undernet~: \url{http://www.undernet.org}
 \item QuakeNet~: \url{http://www.quakenet.org}
 \item Freenode~: \url{http://www.freenode.org}
 \begin{itemize}[<+->]
  \item Alexis Lahouze~: xals
  \item Samuel Thibault~: youpi
  \item Cyril Roelandt~: Steap
  \item \url{\#aquilenet}
  \item \url{\#adsillh}
  \item \url{\#abul}
 \end{itemize}
 \item Geeknode~: \url{http://www.geeknode.org}
 \begin{itemize}[<+->]
  \item Alexis Lahouze~: xals
  \item Samuel Thibault~: youpi
  \item Cyril Roelandt~: Steap
  \item \url{\#fdn}
  \item \url{\#sysnove}
 \end{itemize}
 \item OFTC (Debian)~: \url{http://www.oftc.net}
 \begin{itemize}[<+->]
  \item \url{\#debian-fr}
 \end{itemize}
\end{itemize}
\end{frame}

\begin{frame}{Internet Relay Chat}
\framesubtitle{Bots}
\begin{itemize}
  \item À l'origine créés pour éviter une prise de contrôle par quelqu'un d'autre (takeover) d'un canal ou d'un pseudo en cas d'absence~;
  \pause
  \item Permettent de transférer des informations sur le canal~;
  \pause
  \item Peuvent réagir à des mots (commandes)~;
  \pause
  \item La plupart des réseaux ont leur propre bot de gestion~:
  \begin{itemize}
   \pause
   \item Freenode~: ChanServ et NickServ
   \pause
   \item Geeknode~: C
  \end{itemize}

  \pause
  \item Quelques logiciels de bot connus~:
  \begin{itemize}
   \pause
   \item Supybot~;
   \pause
   \item Hubot~;
   \pause
   \item Eggdrop~;
   \pause
   \item Etc.
  \end{itemize}
 \end{itemize}
\end{frame}
 
\begin{frame}{Internet Relay Chat}
\framesubtitle{Clients les plus connus}
\begin{itemize}
 \item Console~:
 \begin{itemize}
  \item BitchX (semble maintenu car derniers commits dans la journée, mais dernière release le 14 novembre 2014)
  \pause
  \item Irssi (toujours actif, dernière release le 21 septembre dernier)
  \pause
  \item Weechat (très actif)
 \end{itemize}
 \pause
 \item Graphiques~:
 \begin{itemize}
  \item mIRC (Windows, développement toujours actif)~;
  \pause
  \item kvIRC (multiplateforme, anciennement KDE)~;
  \pause
  \item konversation (KDE)~;
  \pause
  \item XChat (GTK)~;
  \pause
  \item Chatzilla (extension Firefox)~;
  \pause
  \item Pidgin (GTK)~;
  \pause
  \item Passerelles XMPP <-> IRC.
 \end{itemize}
\end{itemize}
\end{frame}

\begin{frame}{XMPP (Jabber)}
\framesubtitle{Informations générales}
\begin{itemize}
 \item Créé en 1998, standardisé IETF en 2002 (RFC 3920 à 3923)~;
 \pause
 \item Client-Serveur-Serveur (interconnexion des domaines)~;
 \pause
 \item Orienté discussion un à un (comme ICQ)~;
 \pause
 \item Liste de contacts~;
 \pause
 \item Chiffrage bout en bout avec OTR et PGP.
\end{itemize}
\end{frame}

\begin{frame}{XMPP (Jabber)}
\framesubtitle{Logiciels Clients}
\begin{itemize}
 \item Pidgin (multiplateforme)
 \pause
 \item Gajim (multiplateforme)
 \pause
 \item Psi (multiplateforme)
 \pause
 \item Xabber (Android)
 \pause
 \item Conversations (Android)
 \pause
 \item Kopete (Linux, multiprotocole)
 \pause
 \item Miranda IM (Windows, multiprotocole)
 \pause
 \item Jappix (Web/Ajax)
 \pause
 \item Bitlbee (Linux, serveur IRC, client multiprotocole)
 \pause
 \item Etc.
\end{itemize}
\end{frame}

\begin{frame}{XMPP (Jabber)}
\framesubtitle{Logiciels Serveurs}
\begin{itemize}
 \item Jabberd14 (C)~;
 \pause
 \item Jabberd2 (C)~;
 \pause
 \item eJabberd (erlang)~;
 \pause
 \item Prosody (Lua)~;
 \pause
 \item Tigase (Java)~;
 \pause
 \item Etc.
\end{itemize}
\end{frame}

\begin{frame}{XMPP (Jabber)}
\framesubtitle{Services XMPP connus}
\begin{itemize}
 \item GTalk (Google)
 \pause
 \item im.apinc.org (FR)
 \pause
 \item jabber.im
\end{itemize}

\end{frame}

\begin{frame}{Autres systèmes}
\begin{itemize}
 \item La plupart propriétaires~;
 \pause
 \item Slack~;
 \begin{itemize}
  \item Notion d'équipes~;
  \pause
  \item Possibilité de connecter des applications externes~;
  \pause
  \item Utilisé par les startups.
 \end{itemize}
 \pause
 \item Gitter (github)~;
 \begin{itemize}
  \item Orienté projet.
 \end{itemize}
 \pause
 \item Certains réseaux sociaux~;
 \pause
 \item Etc.
\end{itemize}
\end{frame}

\subsection{Communication asynchrone}

\begin{frame}{Notion de base}
 \begin{itemize}
  \item On sait quand ça part, mais on ne sait pas quand ça arrive.
 \end{itemize}
\end{frame}

\begin{frame}{Usenet/Newsgroups}
\begin{itemize}
 \item Inventé en 1979~;
 \pause
 \item Client-serveur~;
 \pause
 \item Protocole UUCP (Unix to Unix Copy Protocol) puis NNTP (Network News Transfer Protocol)~;
 \pause
 \item Réseau de forums~;
 \pause
 \item Génération, stockage et récupération d'articles~;
 \pause
 \item Très proche du mail~;
 \pause
 \item Notion de groupes et d'abonnements à ces groupes.
\end{itemize}
\end{frame}

\begin{frame}[fragile]{Mails}
\framesubtitle{Informations générales}
\begin{itemize}
 \item Prémisses en 1965~;
 \pause
 \item RFC 561 en 1971, puis RFC 680, 724 et enfin 733 en 1977~: entêtes et protocole~;
 \pause
 \item RFC 822 en 1982~: format des messages~;
 \pause
 \item RFC 2045 à 2049~: MIME~;
 \pause
 \item RFC 2822 en 2001~: remplace la 822~;
 \pause
 \item RFC 5322 en 2008~: étend la 2822~;
 \pause
 \item Client-Serveur-Serveur~;
 \pause
 \item Chiffrage avec PGP (clé asymétrique), ou S/MIME (certificat numérique)~;
 \pause
 \item Organisation par sujets (threads)~:
 \begin{itemize}
  \pause
  \item Entête \verb/In-Reply-To/.
 \end{itemize}
\end{itemize}
\end{frame}

\begin{frame}{Mails}
\framesubtitle{Mailing-lists}
\begin{itemize}
 \item Mails gérés par un robot de liste, ou gestionnaire de listes~;
 \pause
 \item Le gestionnaire permet de gérer les abonnements~;
 \pause
 \item Adresse de destination unique~;
 \pause
 \item Redirection à tous les abonnés~;
 \pause
 \item Utilisées pour communiquer entre les membres de l'équipe~;
 \pause
 \item Utilisées pour communiquer avec l'extérieur (newsletters)~;
 \pause
 \item Souvent couplées à une interface web de gestion~;
 \pause
 \item Archives~;
 \pause
 \item Aggrégateurs de listes publiques~:
 \begin{itemize}
 \pause
  \item \url{http://www.gmane.org} (listes de diffusion et newsgroups)~;
 \end{itemize}

 \pause
 \item Moteurs connus~:
 \begin{itemize}
  \pause
  \item Sympa
  \pause
  \item Mailman
  \pause
  \item Majordomo
 \end{itemize}
\end{itemize}
\end{frame}

\begin{frame}{Mails}
\framesubtitle{Mailing-lists, entêtes spécifiques}
\begin{description}
 \item RFC 4021~;
 \pause
 \item [List-ID] Identifiant de la liste, permet une classification~;
 \pause
 \item [List-Owner] Adresse du propriétaire de la liste~;
 \pause
 \item [List-Post] Adresse d'envoi de la liste~;
 \pause
 \item [List-Subscribe] Adresse d'abonnement à la liste~;
 \pause
 \item [List-Unsubscribe] Adresse de désabonnement de la liste~;
 \pause
 \item [List-Help] Adresse pour obtenir de l'aide sur la liste~;
 \pause
 \item [List-Archive] Adresse de l'archive de la liste.
\end{description}
\end{frame}

\begin{frame}[fragile]{Forum}
\begin{itemize}[<+->]
 \item Site web~;
 \pause
 \item Hierachie~: Catégories, Sujets, Messages~;
 \pause
 \item Souvent des passerelles mails (envoi et gestion des réponses)~:
 \begin{itemize}[<+->]
  \item catégorie => mailing list~;
  \pause
  \item sujet => thread~;
  \pause
  \item message => mail.
 \end{itemize}
 \pause
 \item \url{https://www.hardware.fr}
\end{itemize}
\end{frame}

\begin{frame}{Bugtrackers}
\begin{itemize}
 \item Site web, mailing lists
 \pause
 \item Permet de recencer et suivre les tâches du projet
 \pause
 \item Anomalies (bugs) et fonctionnalités
 \pause
 \item Moteurs connus~:
 \begin{itemize}
  \item Bugzilla~;
  \pause
  \item Mantis~;
  \pause
  \item Flyspray~;
  \pause
  \item Debian BTS (emails, chaque bug est une liste de diffusion)~;
  \pause
  \item Forges.
 \end{itemize}
\end{itemize}
\end{frame}
