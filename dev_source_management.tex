\subsection{Gestion du code source}

\begin{frame}{Systèmes de gestion de versions}
\begin{itemize}[<+->]
 \item Local~:
 \begin{itemize}[<+->]
  \item SCCS~: remplacé par~:
  \item RCS~: remplacé par~:
 \end{itemize}
 \item Client-serveur~:
 \begin{itemize}[<+->]
  \item Tous ceux utilisés actuellement.
 \end{itemize}
 \item Deux modes principaux~:
 \begin{itemize}
 \item Centralisé
 \item Décentralisé
 \end{itemize}
\end{itemize}
\end{frame}

\begin{frame}{Systèmes de gestion de versions centralisés}
\begin{itemize}[<+->]
 \item Principe~:
 \begin{itemize}
  \item Chaque modification (commit) est envoyée directement sur le serveur.
 \end{itemize}
 \item Les plus connus~:
 \begin{itemize}[<+->]
  \item CVS
  \item Subversion
 \end{itemize}
 \item Inconvénient~: Nécessite une connexion pour envoyer ses modifications.
\end{itemize}
\end{frame}

\begin{frame}{Systèmes de gestion de versions décentralisés}
\begin{itemize}[<+->]
 \item Principe~:
 \begin{itemize}
  \item Chaque modification (commit) est stockée localement.
  \item L'ensemble des modifications est envoyée à un dépôt distant (remote)~;
  \item Possibilité d'avoir plusieurs dépôts distants.
 \end{itemize}
 \item Les plus connus~:
 \begin{itemize}[<+->]
  \item BitKeeper
  \item Git
  \item Mercurial
  \item GNU Bazaar (successeur de GNU Arch)
 \end{itemize}
 \item Avantage~: Pas besoin de connexion pour commiter~;
 \item Inconvénient~: Workflow plus complexe.
\end{itemize}
\end{frame}

\begin{frame}{Lexique}
\begin{description}[align=left]
 \item [Trunk] version de développement principale~;
 \begin{itemize}[<+->]
  \item Très instable~;
  \item Version à partir de laquelle on crée les branches et les tags~;
  \item Appelé aussi 'master' (Git).
 \end{itemize}
 \pause
 \item [Tags] version figée (release)~;
 \begin{itemize}[<+->]
  \item Ne doit pas être modifié directement\dots
  \item \dots donc utilisation de branches de maintenance.
 \end{itemize}
 \pause
 \item [Branche] version de développement secondaire~;
 \begin{itemize}[<+->]
  \item Branche de maintenance de version (hotfix)~;
  \item Branche de fonctionnalité (feature).
 \end{itemize}
 \pause
 \item [Révision/commit] modification atomique d'un ou plusieurs fichiers~;
 \pause
 \item [Head] dernière révision du dépôt~;
 \pause
 \item [Objet] élément versionnable (fichier, répertoire, lien symbolique, etc.).
\end{description}
\end{frame}

\begin{frame}{Utilisation générale}
\framesubtitle{Que faut-il versionner~?}
\begin{itemize}[<+->]
 \item Tout ce qui est susceptible d'être modifié au cours du temps~:
 \begin{itemize}[<+->]
  \item Code source~;
  \item Script de construction~;
  \item Documentation.
 \end{itemize}
 \item Mais pas ce qui est générable~:
 \begin{itemize}[<+->]
  \item Code compilé~;
  \item Documentation d'API~;
  \item Etc.~;
  \item Les différents systèmes possèdent un mécanisme d'ignore.
 \end{itemize}
\end{itemize}
\end{frame}

\begin{frame}{Utilisation générale}
\framesubtitle{Quand faut-il commiter~?}
\begin{itemize}[<+->]
 \item Le plus souvent possible~;
 \item Quand ça marche~:
 \begin{itemize}[<+->]
  \item Compile sans erreur~;
  \item Testé et validé.
 \end{itemize}
 \item Un commit inclut une modification et une seule.
 \begin{itemize}[<+->]
  \item Éviter les commits monolythiques.
 \end{itemize}
\end{itemize}
\end{frame}

\begin{frame}[fragile]{Subversion}
\framesubtitle{Informations générales}
\begin{itemize}[<+->]
 \item Anciens développeurs de CVS~;
 \item Première version en 2000\dots
 \item \dots version 1.0.0 en 2004~;
 \item Dernière version~:1.9.5 (29 novembre 2016)~;
 \item Un dépôt central, sur un serveur, sur les protocoles~:
 \begin{itemize}[<+->]
  \item SVN (svnserve, TCP 3690)~;
  \item HTTP (webdav, passe dans plus d'infrastructures réseau).
 \end{itemize}
 \item Pas de notion de branche ou de tag, utilisation de l'arborescence~;
 \item Possibilité de checkout une sous-arborescence~;
 \item Numéros de versions séquentiels, globaux au dépôt~;
 \item Les répertoires sont versionnés (notamment les répertoires vides)~;
 \item \url{http://svnbook.red-bean.com/} (assez ancienne, mais toujours d'actualité).
\end{itemize}
\end{frame}

\begin{frame}{Subversion}
\framesubtitle{Architecture d'un dépôt}
  \begin{itemize}[<+->]
   \item trunk
   \begin{itemize}[<+->]
    \item Développements courants~;
   \end{itemize}
   \item branches
   \begin{itemize}[<+->]
    \item Développements conséquents~;
    \item Maintenances de versions~;
   \end{itemize}
   \item tags
   \begin{itemize}[<+->]
    \item Versions figées~;
    \item Pas de modification dans les tags~;
   \end{itemize}
 \end{itemize}
\end{frame}

\begin{frame}{Subversion}
\framesubtitle{Architecture d'un dépôt}
 \begin{figure}
   \caption{Worflow de développement avec Subversion.}
   \includegraphics[width=345pt]{repository-structure3}\par
   \tiny Source~: \url{http://blogs.wandisco.com/2011/10/24/subversion-best-practices-repository-structure/}
 \end{figure}
\end{frame}

\begin{frame}{Subversion}
\framesubtitle{Workflow de développement}
\begin{description}
 \item [\Q{svn checkout}] créer sa copie locale~;
 \pause
 \item [\Q{svn update}] mettre à jour sa copie locale avec les modifications des autres~;
 \pause
 \item [\Q{svn add}] ajouter des fichiers~;
 \pause
 \item [\Q{svn copy}] copier des fichiers ou une sous-arborescence~;
 \begin{itemize}
  \item Sert particulièrement à créer les branches et les tags.
 \end{itemize}
 \pause
 \item [\Q{svn delete}] supprimer des fichiers~;
 \pause
 \item [\Q{svn move}] déplacer des fichies~;
 \pause
 \item [\Q{svn status}] voir l'état de la copie locale~;
 \pause
 \item [\Q{svn diff}] voir les différences entre deux versions, ou de sa copie locale par rapport à la dernière mise à jour du dépôt~;
 \pause
 \item [\Q{svn revert}] annuler une modification application inversée du diff d'un commit)~;
 \pause
 \item [\Q{svn resolve}] effectuer une gestion des conflits sur un fichier~;
 \pause
 \item [\Q{svn resolved}] marquer un conflit comme résolu~;
 \pause
 \item [\Q{svn commit}] pousser ses modifications sur le dépôt~;
 \pause
 \item [\Q{svn merge}] fusionner une branche dans une autre~;
 \pause
 \item [\Q{svn switch}] changer de branche (changer le point d'origine dans l'arborescence distante)~;
 \pause
 \item [\Q{svn info}] avoir les informations de sa copie locale.
\end{description}
\end{frame}

\begin{frame}[fragile]{Subversion}
\frametitle{svn status}
\begin{itemize}[<+->]
 \item Affiche l'état de la copie locale~;
 \item Permet d'avoir les modifications sur le serveur avec l'option \verb/--show-updates/ ou \verb/-u/~;
\end{itemize}
\begin{snvlisting}{Exemple d'invocation (honteusement copié sur \url{http://svnbook.red-bean.com/en/1.8/svn.ref.svn.c.status.html})}[][escapeinside={}]
# svn status -u wc
 M            965    wc/bar.c
        *     965    wc/foo.c
A  +          965    wc/qax.c
Status against revision:    981
\end{snvlisting}
\end{frame}

\begin{frame}[fragile]{Subversion}
\frametitle{svn status -- format}
\begin{itemize}[<+->]
 \item Première colonne~: changements sur l'objet~:
 \begin{description}[align=left]
  \item [\textvisiblespace] objet inchangé~;
  \item [A] objet ajouté (\verb/svn add/)~;
  \item [D] objet supprimé (\verb/svn delete/)~;
  \item [M] objet modifié~;
  \item [C] objet en conflit (après \verb/svn update/)~;
  \item [I] objet ignoré~;
  \item [?] objet inconnu (non ajouté)~;
  \item [!] objet manquant (supprimé du système de fichiers mais pas avec \verb/svn delete/)
  \item [\textasciitilde] objet ayant changé de type (par ex.~: fichier -> répertoire)
 \end{description}
\end{itemize}
\end{frame}

\begin{frame}[fragile]{Subversion}
\frametitle{svn status -- format}
\begin{itemize}[<+->]
 \item Seconde colonne~: changements sur les propriétés de l'objet~:
 \begin{description}[align=left]
  \item [\textvisiblespace] propriétés inchangées~;
  \item [M] propriétés modifiées~;
  \item [C] propriétés en conflit (modifiées localement et sur le dépôt distant).
 \end{description}
 \item Troisième colonne~: L si copie locale verrouillée~;
 \item Quatrième colonne~: + si ajout avec historique~;
 \item Cinquième colonne~: S si l'objet a changé de branche~;
 \item Sixième colonne~: informations de verrouillage distant~;
 \item Septième colonne~: informations de conflit d'arbre~;
 \item Huitième colonne~: toujours une espace~;
 \item Neuvième colonne~: * si l'objet est périmé (mis à jour sur le dépôt distant)~;
 \item Révision locale (si \verb/--show-updates/ ou \verb/--verbose/ passé en paramètre)~;
 \item Dernière révision commitée (si \verb/--verbose/ passé en paramètre)~;
 \item Toujours en dernier~: chemin de l'objet concerné, peut contenir des espaces.
\end{itemize}
\end{frame}

\begin{frame}[fragile]{Subversion}
\frametitle{svn info}
\begin{snvlisting}{Exemple d'invocation (honteusement copié sur \url{http://svnbook.red-bean.com/en/1.8/svn.ref.svn.c.info.html})}[][escapeinside={}]
# svn info foo.c
Path: foo.c
Name: foo.c
Working Copy Root Path: /home/sally/projects/test
URL: http://svn.red-bean.com/repos/test/foo.c
Repository Root: http://svn.red-bean.com/repos/test
Repository UUID: 5e7d134a-54fb-0310-bd04-b611643e5c25
Revision: 4417
Node Kind: file
Schedule: normal
Last Changed Author: sally
Last Changed Rev: 20
Last Changed Date: 2003-01-13 16:43:13 -0600 (Mon, 13 Jan 2003)
Text Last Updated: 2003-01-16 21:18:16 -0600 (Thu, 16 Jan 2003)
Properties Last Updated: 2003-01-13 21:50:19 -0600 (Mon, 13 Jan 2003)
Checksum: d6aeb60b0662ccceb6bce4bac344cb66
\end{snvlisting}
\end{frame}

\begin{frame}[fragile]{Subversion}
\frametitle{Propriétés}
\begin{itemize}[<+->]
 \item Métadonnées des objets~;
 \item Clé-valeur~;
 \item Personnalisées ou réservées à subversion (préfixées par \verb/svn:/)
 \item Commandes~:
 \begin{description}
  \item [\Q{svn proplist <objet>}] affiche les propriétés d'un objet~;
  \item [\Q{svn propedit <nom> <objet>}] ouvre un éditeur pour définir la valeur d'une propriété d'un objet~;
  \item [\Q{svn propset <nom> <valeur> <objet>}] définit la valeur d'une propriété d'un objet~;
  \item [\Q{svn propdel <nom> <objet>}] supprime la propriété de l'objet.
 \end{description}
 \item Quelques propriétés connues~:
 \begin{description}
  \item [\Q{svn:ignore}] liste des objets non versionnés~;
  \item [\Q{svn:eol-style}] type de fin de ligne (unix, dos)~;
  \item [\Q{svn:author}] auteur de la révision~;
  \item [\Q{svn:date}] date de la révision~;
  \item [\Q{svn:log}] message de révision.
 \end{description}
\end{itemize}
\end{frame}

\begin{frame}[fragile]{Subversion}
\frametitle{\Q{svn:ignore}}
\begin{itemize}[<+->]
 \item Contient la liste des patterns d'objets ignorés~;
 \item S'applique à un répertoire.
\end{itemize}
\begin{snvlisting}{Exemple}[][escapeinside={}]
*~
.*.sw*
*.pyc
\end{snvlisting}
\end{frame}

\begin{frame}{Subversion}
 \framesubtitle{Outils annexes}
 \begin{description}
  \item [TortoiseSVN] Intégration dans l'explorateur de fichiers Windows~;
  \pause
  \item [eSVN] Interface graphique dédiée, multiplateforme~;
  \pause
  \item [kdesvn] Interface graphique dédiée, Linux uniquement~;
  \pause
  \item [Différents IDE] La plupart des IDE ont une intégration de subversion~;
  \pause
  \item [Forges] Outils web intégrés pour le développement.
  \pause
  \item [Etc.] Modules d'éditeurs de texte, intégration gestionnaires de fichiers, \dots
  \pause
 \end{description}
\end{frame}

\begin{frame}[fragile]{Git}
\frametitle{Informations générales}
\begin{itemize}[<+->]
 \item Initialement écrit par Linus Torvalds pour remplacer BitKeeper~;
 \item Première version le 7 avril 2005~;
 \item Dernière version le 4 août 2017 (2.14.1)~;
 \item Possibilité d'avoir plusieurs dépôts distants (remote)~;
 \item Accessibles via les protocoles~:
 \begin{itemize}
  \item SSH (authentification par clé)
  \item HTTP (clonage anonyme)
  \item git (en pratique peu utilisé)
 \end{itemize}
 \item La copie locale est un clone (complet ou incomplet) du dépôt distant~;
 \item Notions de tags et de branches~;
 \item Pas de séquence, utilisation de hash pour identifier les commits~;
 \item Les répertoires ne sont pas versionnés, donc absence de répertoires vides~;
 \item Possibilité d'étendre la commande (\verb/git flow/ par exemple)~;
 \item \url{https://git-scm.com/book/fr/v2/}.
\end{itemize}
\end{frame}

\begin{frame}{Git}
\framesubtitle{Terminologie}
\begin{description}
 \item [blob] contenu d'un objet versionné (fichier)~;
 \pause
 \item [commit] révision, hash et commit parent~;
 \pause
 \item [branche] version active du projet~;
 \pause
 \item [tag] étiquette sur un point donné de l'historique, permet de figer des releases~;
 \pause
 \item [merge] fusion entre deux branches~;
 \pause
 \item [rebase] changement de la base de la branche par une autre branche~;
 \pause
 \item [staging] aussi appelé cache, contient les objets et états servant au commit~;
 \pause
 \item [stash] pile dans laquelle le développeur peut stocker ses modifications sans les commiter, souvent avant un pull pour fusionner ses modifications a posteriori.
\end{description}
\end{frame}

\begin{frame}{Git}
\framesubtitle{Workflow}
\begin{description}
 \item [\Q{git clone}] clone un dépôt distant en local~;
 \pause
 \item [édition] modifie les fichiers~;
 \pause
 \item [\Q{git stash}] sauvegarde les modifications dans stash~;
 \pause
 \item [\Q{git pull}] récupére les révisions distantes~;
 \pause
 \item [\Q{git stash pop}] réapplique les modifications sur la copie de travail à jour~;
 \pause
 \item [\Q{git mergetool}] lance l'outil de gestion des conflits (lors d'un pull, ou d'un merge)~;
 \pause
 \item [\Q{git add}] ajoute un fichier modifié ou non versionné dans le cache~;
 \pause
 \item [\Q{git rm}] supprime un fichier et le met comme tel dans le cache~;
 \pause
 \item [\Q{git commit}] crée une révision locale d'après le cache~;
 \pause
 \item [\Q{git push}] pousse les révisions locales~;
\end{description}
\end{frame}
 
\begin{frame}{Git}
\framesubtitle{Workflow -- suite}
\begin{description}
 \item [\Q{git branch}] manipule les branches~;
 \pause
 \item [\Q{git checkout}] change de branche, ou restaure un fichier de travail~;
 \pause
 \item [\Q{git reset}] réinitialise HEAD à un état différent~;
 \pause
 \item [\Q{git merge}] fusionne une branche dans la branche courante~;
 \pause
 \item [\Q{git mergetool}] lance l'outil de gestion des conflits si nécessaire~;
 \pause
 \item [\Q{git rebase}] reconstruit l'historique de la branche par rapport à une autre branche~;
 \pause
 \item [\Q{git tag}] manipule les tags, possibilité de le signe avec une clé PGP.
\end{description}
\end{frame}

\begin{frame}[fragile]{Git}
\frametitle{Autres commandes utiles}
\begin{description}
 \pause
 \item [\Q{git blame}] affiche le commit et l'auteur de chaque ligne d'un fichier~;
 \pause
 \item [\Q{git diff}] affiche les différences sur la copie locale, ou entre deux révisions~;
 \begin{itemize}[<+->]
  \item Permet de générer un fichier patch, avec l'option \verb/--patch/ ou \verb/-p/ ou encore \verb/-u/~;
 \end{itemize}
 \pause
 \item [\Q{git submodule}] gère les sous-modules~;
 \pause
 \item [\Q{git help <command>}] RTFM ;-) Attention, très (trop) complet.
\end{description}
\end{frame}

\begin{frame}{Git}
 \frametitle{Workflow simple}
 \begin{figure}
   \caption{Worflow de développement local avec Git.}
   \includegraphics[height=160pt]{git-base-workflow}\par
   \tiny Source~: \url{http://tex.stackexchange.com/questions/70320/workflow-diagram}
 \end{figure}
\end{frame}

\begin{frame}[fragile]{Git}
\frametitle{\Q{git status}}
\begin{snvlisting}{Exemple}[][escapeinside=||]
On branch master
Your branch is up-to-date with 'origin/master'.
Changes to be committed:
  (use "git reset HEAD <file>..." to unstage)

        modified:   mailcap

Changes not staged for commit:
  (use "git add <file>..." to update what will be committed)
  (use "git checkout -- <file>..." to discard changes in working directory)

        modified:   caffrc
        modified:   config/i3pystatus/config.py
        modified:   dotfilesrc
        modified:   mutt-profiles/alexis@lahouze.org/profile
        modified:   mutt-profiles/alexis@sysnove.fr/profile
        modified:   offlineimaprc
        modified:   services/dunst/log/run
        modified:   tmux.conf
        modified:   vimrc
        modified:   zshrc

Untracked files:
  (use "git add <file>..." to include in what will be committed)

        config/flexget/
        imapnotify.js

\end{snvlisting}
\end{frame}

\begin{frame}[fragile]{Git}
\frametitle{Sous-modules}
\begin{itemize}[<+->]
 \item Référence de dépôts git dans des sous-répertoires du dépôt~;
 \item Ne se mettent pas à jour automatiquement~;
 \item Enregistré dans \verb/.gitmodules/~;
 \item Workflow~:
 \begin{itemize}[<+->]
  \item \verb/git submodule add/
  \item \verb/git commit -m "..."/
  \item \verb/git submodule update/
 \end{itemize}
 \item \url{https://git-scm.com/book/fr/v1/Utilitaires-Git-Sous-modules}
\end{itemize}
\end{frame}

\begin{frame}[fragile]{Git}
 \frametitle{Git Flow}
 \begin{itemize}[<+->]
  \item Décrit par Vincent Driesse sur \url{http://nvie.com/posts/a-successful-git-branching-model/}
  \item Process de développement et de publication standardisé~;
  \item Commande \verb/git flow/ (\url{https://github.com/nvie/gitflow}).
  \item Antisèche~: \url{http://danielkummer.github.io/git-flow-cheatsheet/}
 \end{itemize}
\end{frame}

\begin{frame}[fragile]{Git}
 \frametitle{Git Flow}
 \begin{figure}
   \caption{Workflow complet avec git flow.}
   \includegraphics[height=160pt]{Gitflow-workflow}\par
   \tiny Source~: \url{https://www.atlassian.com/git/tutorials/comparing-workflows/gitflow-workflow}
 \end{figure}
\end{frame}

\begin{frame}[fragile]{Git}
\frametitle{git svn}
\begin{itemize}[<+->]
 \item Possibilité pour Git de se connecter à un dépôt Subversion~;
 \item Permet d'effectuer des commits locaux~;
 \item Ne permet pas de décentraliser le développement.
 \item \url{https://git-scm.com/book/fr/v2/Git-et-les-autres-systèmes-Git-comme-client#Git-et-Subversion}
\end{itemize}
\end{frame}

\begin{frame}[fragile]{Git}
\frametitle{git svn -- utilisation de base}
\begin{description}
 \item [\Q{git svn clone}] équivalent à \verb/svn checkout/~;
 \pause
 \item [\Q{git add}, \Q{git commit}] travailler localement~;
 \pause
 \item [\Q{git svn rebase}] équivalent à \verb/svn update/~;
 \pause
 \item [\Q{git svn dcommit}] équivalent à \verb/svn commit/ pour chaque commit local~;
 \pause
 \item [\Q{git svn branch}] manipule les branches et les tags (\verb/--tag/ ou \verb/--t/)~;
 \pause
 \item [\Q{git svn log}] affiche l'historique~;
 \pause
 \item [\Q{git svn blame}] affiche l'auteur et la révision de chaque ligne d'un fichier~;
 \pause
 \item [\Q{git svn reset}] permet de revenir à une révision spécifique~;
 \pause
 \item [\Q{git svn help}] RTFM ;-)
\end{description}
\end{frame}

\begin{frame}[fragile]{Git}
 \frametitle{Exemples - Cas simple - initialisation}
 \begin{itemize}[<+->]
  \item Créer un dépôt sur Github
  \item Initialiser la copie locale: \verb/git init/
  \item Configurer son git local (user, email): \verb/git config  user.name 'Alexis Lahouze'/ \verb/git config user.email 'alexis@sysnove.fr'/
  \item Ignorer les fichiers constructibles, temporaires, etc. \verb/vim .gitignore/ (rajouter par exemple \verb/.*.sw?/, \verb/*.log/, \verb/*.out/, etc \dots)
  \item Ajouter les fichiers nécessaires au projet: \verb\git add *.kilepr *.tex imgs/* *.txt\
  \item Commiter la branche \verb/master/ : \verb\git commit -m "Initialize repository."\
 \end{itemize}
\end{frame}

\begin{frame}[fragile]{Git}
 \frametitle{Exemples - Cas simple - développement dans branche develop }
 \begin{itemize}[<+->]
  \item Créer une branche \verb/develop/ et basculer dessus: \verb\git checkout -b develop\
  \item Ajouter un fichier README: \verb\vim README.md\
  \item Ajouter dans le cache: \verb\git add README.md\
  \item Commiter: \verb\git commit -m "Add readme."\
  \item Pousser la branche develop: \verb\git push -u origin develop\
  \item Modifier un fichier: \verb\vim forges.tex\
  \item Ajouter dans le cache: \verb\git add -p forges.tex\
  \item Commiter: \verb\git commit -m "Improve forges.tex"\
  \item Pousser à nouveau: \verb\git push\
  \item Merger dans \verb/master/ : \verb\git checkout master\ , \verb\git merge develop\
 \end{itemize}
\end{frame}

\begin{frame}[fragile]{Git}
 \frametitle{Exemples - Cas simple - Remisage }
 \begin{itemize}[<+->]
  \item Modifier un fichier, par exemple README.md
  \item Remiser les modifications: \verb\git stash\
  \item Afficher les données remisées: \verb\git stash list\
  \item Afficher un remisage particulier: \verb/git stash show stash@\{1\}/
  \item Récupérer les modifications distantes: \verb\git pull\
  \item Réappliquer les modifications remisées: \verb\git stash pop\
  \item Éventuellement éditer les conflits: \verb\git mergetool\
  \item Valider la gestion du conflit: \verb\git commit --no-edit\
  \item Ajouter les modifications du fichier README.md: \verb\git add -p README.md\
  \item Pousser toutes les modifications: \verb\git push\
 \end{itemize}
\end{frame}

\begin{frame}[fragile]{Git}
 \frametitle{Exemples - Cas simple - Rebase }
 \begin{itemize}[<+->]
  \item Importer les modifications de la branche parente: \verb\git rebase\
  \item Editer les conflits si nécessaire: \verb\git mergetool\
  \item Continuer le rebasage: \verb\git rebase --continue\
  \item Afficher l'état du dépôt: \verb\git status\
 \end{itemize}
\end{frame}

\begin{frame}[fragile]{Git}
 \frametitle{Exemples - Git flow }
 \begin{itemize}[<+->]
  \item Installer git flow sous Debian ou Ubuntu: \verb\sudo apt install git-flow\
  \item Initialiser git flow dans le dépôt: \verb\git flow init\
  \item Répondre aux questions (production release: \verb\master\, next release: \verb\develop\, etc \dots)
  \item Créer une nouvelle branche de feature: \verb\git flow feature start mafeature\
  \item Vérifier les branches présentes: \verb\git branch\
  \item Éditer, commiter, etc.
  \item Publier pour que d'autres participent au développement: \verb\git flow feature publish\
  \item Récupérer les modifications des autres: \verb\git pull\
  \item Récupérer les dernières features et corrections terminées: \verb\git flow feature rebase\
  \item Terminer la feature: \verb\git flow feature finish\
 \end{itemize}
\end{frame}
